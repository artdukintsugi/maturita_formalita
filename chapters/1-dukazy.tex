%chktex-file 44

\mathquote{Only prove something that already seems to be obvious.}{Grant Sanderson}


V této kapitole se seznámíme s některými základními pojmy a pravidly, které nám pomohou v matematickém uvažování. Základními bloky jsou pojem výrok a práce s nimi. Po zavedení těchto pojmů se seznámíme s důkazy, díky kterým můžeme ověřit správnost tvrzení.

\section{Výroková logika}

Zde se budeme pouze věnovat základní výrokové logice a na konci si lehce zmíníme něco o predikátové logice. Výroková logika je základním nástrojem matematiky, který nám umožňuje formulovat veškeré matematické tvrzení.

\begin{tcolorbox}
  Výrok je tvrzení u kterého lze rozhodnout, zda je pravdivé nebo nepravdivé. Výroky budeme značit písmeny $p, q, r, \ldots$. Pravdivost výroku budeme značit $1$ a nepravdivost $0$. Těmto číslům se říká pravdivostní hodnoty.
\end{tcolorbox}
\begin{example}
  Uvedeme si zde několik příkladů výroků:
  \begin{itemize}
    \item $p$: Venku prší.
    \item $q$: $2 + 2 = 69$.
    \item $r$: Praha je hlavní město České republiky.
    \item $s$: $4$ je prvočíslo.
    \item $t$: $x^2 + 1 = 0$ má reálná řešení.
  \end{itemize}
\end{example}

\begin{problem}
  Koukněte se na příklad~\ref{ex:1.1.1} a rozhodněte, zda jsou jednotlivé výroky pravdivé nebo nepravdivé.
\end{problem}

\begin{solution}
  \begin{itemize}
    \item $p$: Venku prší. --- Špatně zapsaný výrok. Není určeno, kde se nacházíme.
    \item $q$: $2 + 2 = 69$. --- Nepravda. 0.
    \item $r$: Praha je hlavní měs to České republiky. --- Pravda. 1.
    \item $s$: $4$ je prvočíslo. --- Pravda. 1.
    \item $t$: $x^2 + 1 = 0$ má reálná řešení. --- Nepravda. 0.
  \end{itemize}
\end{solution}

\pagebreak
\begin{example}
  Důležité je také uvést si, co výrok není. 
  \begin{itemize}
    \item Jak se máš?
    \item Bum!
    \item Pozor na psa!
    \item Černá je nejlepší barva. (Toto je názor, nikoliv výrok. Redakce si avšak myslí, že je to pravda.)
  \end{itemize}
\end{example}

Ještě než se vrhneme na logické spojky, představíme si co to znamená negace výroku.

\begin{tcolorbox}
  \textbf{Negace} nebo-li NOT výroku $p$ je výrok, který tvrdí opak výroku $p$. Negaci výroku $p$ budeme značit $\neg p$.
\end{tcolorbox}

\begin{example}
  Vezměme si výrok $p$: Voda v moři je slaná. Negací tohoto výroku je výrok $\neg p$: Voda v moři není slaná. Upozorňujeme, že opravdu měníme jen slovo `je' na `není'. Nepožíváme například `Voda v moři je sladká'.
\end{example}

\begin{problem}
  Opět se vraťte k příkladu~\ref{ex:1.1.1} a určete negace jednotlivých výroků.
\end{problem}

\begin{solution}
  \begin{itemize}
    \item $\neg p$: Venku neprší.
    \item $\neg q$: $2 + 2 \neq 69$.
    \item $\neg r$: Praha není hlavní město České republiky.
    \item $\neg s$: $4$ není prvočíslo.
    \item $\neg t$: $x^2 + 1 = 0$ nemá reálná řešení.
  \end{itemize}
\end{solution}

Výroky samotné jsou sice fajn ale co s nimi dělat? K tomu nám poslouží logické spojky. Je to způsob jakým můžeme spojovat výroky do složitějších celků. Proč? Dozvíte se za chvíli.

\begin{tcolorbox}
  Rozlišujeme 4 základní logické spojky:

  \textbf{Konjukce ($\land$):} Konjukce nebo-li AND výroků $p$ a $q$ je výrok, který je pravdivý právě tehdy, když jsou pravdivé oba výroky $p$ a $q$. 

  \textbf{Disjunkce ($\lor$):} Disjunkce nebo-li OR výroků $p$ a $q$ je výrok, který je pravdivý právě tehdy, když je pravdivý alespoň jeden z výroků $p$ a $q$.

  \textbf{Implikace ($\Rightarrow$):} Implikace nebo-li IF-THEN výroků $p$ a $q$ je výrok, který je nepravdivý právě tehdy, když je pravdivý výrok $p$ a nepravdivý výrok $q$. V tomto případě lze nad tím přemýšlet jako nad slibem. Když mi někdo něco slíbí, tak to musí splnit, ale pokud to neslíbí, tak je jedno co udělá.

  \textbf{Ekvivalence ($\Leftrightarrow$):} Ekvivalence nebo-li IF-AND-ONLY-IF výroků $p$ a $q$ je výrok, který je pravdivý právě tehdy, když jsou oba výroky $p$ a $q$ pravdivé nebo oba nepravdivé.

  \vspace{0.2cm}

  Pro větší přehlednost si můžete vytvořit tabulku pravdivostních hodnot pro jednotlivé logické spojky.

  \begin{center}
    \begin{tabular}{|c|c|c|c|c|c|c|}
      \hline
      $p$ & $q$ & $p \land q$ & $p \lor q$ & $p \Rightarrow q$ & $p \Leftrightarrow q$ \\
      \hline
      0 & 0 & 0 & 0 & 1 & 1 \\
      0 & 1 & 0 & 1 & 1 & 0 \\
      1 & 0 & 0 & 1 & 0 & 0 \\
      1 & 1 & 1 & 1 & 1 & 1 \\
      \hline
    \end{tabular}
  \end{center}
\end{tcolorbox}

\pagebreak

\begin{example}
  Uveďme si několik příkladů na logické spojky:
  \begin{itemize}
    \item Konjukce: $p \land q$: Venku prší a je teplo.
    \item Disjunkce: $p \lor q$: Venku prší nebo je teplo.
    \item Implikace: $p \Rightarrow q$: Když prší, tak je mokro.
    \item Ekvivalence: $p \Leftrightarrow q$: Venku prší právě tehdy, když je mokro.
  \end{itemize}
\end{example}

\begin{problem}
  Vraťte se k příkladu~\ref{ex:1.1.1} a zkuste si spojit výroky pomocí logických spojek.
\end{problem}

\section{Odbočka --- Predikátová logika}
Predikátová logika staví na výrokové logice a rozšiřuje ji o kvantifikátory. Ty nám umožňují mluvit o všech prvcích nějaké množiny. Základními kvantifikátory jsou:
\begin{itemize}
  \item Univerzální kvantifikátor $\forall$: Výrok $\forall x \in M: P(x)$ znamená, že výrok $P(x)$ platí pro všechny prvky množiny $M$.
  \item Existenciální kvantifikátor $\exists$: Výrok $\exists x \in M: P(x)$ znamená, že existuje prvek množiny $M$, pro který platí výrok $P(x)$.
\end{itemize}

Toto se nám bude hodit při důkazech, kde budeme chtít mluvit o všech prvcích nějaké množiny (co je množina se také dozvíte). Negace a spojování kvantifikátorů také funguje, ale je to lehce složitější.

\section{Odbočka --- Přirozená dedukce}
Jeden z formálních dokazovacích systémů je přirozená dedukce. Tento systém je založen na pravidlech, které nám umožňují odvodit nové výroky z již známých. Většina pravidel je intuitivních a vychází z toho, jak my lidé přirozeně uvažujeme.


