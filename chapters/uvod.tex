
\mathquote{Sine mathematica vita nulla est (Bez matematiky život nestojí za nic).}{autor neznámý}

Tento text má být snaha o trochu pokročilejší nahled na elementární matematiku potřebnou k maturitě. Měl by být vhodný jak pro studenty, kteří se chtějí připravit na maturitu, tak pro ty, kteří chtějí získat hlubší porozumění matematice. Než se pustíme do vysvětlení jak tato kniha funguje, chci Vás odkázat opět na můj email, kam můžete zasílat veškeré dotazy, nejasnosti a připomínky: 
\begin{center}
  \textbf{\href{mailto: jonas.vybiral.jv@gmail.com}{jonas.vybiral.jv@gmail.com}}.
\end{center}
Bežné matematické knihy jsou psané ve formátu definice-věta-důkaz (co tyto věci znamenají se dozvíte brzy). Toto však v tomto textu naleznete pouze v omezené míře. Místo toho se budeme snažit vysvětlit matematické pojmy a věty tak, aby byly srozumitelné pro každého. Pokusím se avšak přidat nějaký nahled na to, jak by to mohlo vypadat, kdybychom se snažili být formální.

Formát knihy bude většinou následující: nejdříve se seznámíme s nějakým pojmem což bude většinou vyznačeno následujícím rámečkem:
\begin{definitionbox}
  Důležitý pojem nebo pravidlo. Rámečkům věnujte extra pozornost.
\end{definitionbox}
Poté čtenáři představíme nějaký příklad, který by měl ilustrovat, jak daný pojem funguje. Příklady budou vyznačeny následovně:
\begin{example}
  Příklad na procvičení daného pojmu. Tento příklad bude i řešený.
\end{example}
\begin{problem}
  Cvičení na procvičení daného pojmu. 
\end{problem}

\begin{hint}
  Pokud bude cvičení náročnější budeme se snažit vždy poskytnout nápovědu aby čtenáři mohla být cesta k řešení usnadněna.
\end{hint}

\begin{solution}
  Řešení cvičení. Občas bude vynecháno, aby si čtenář mohl cvičení zkusit sám. Vynechávat řešení budeme většinou jen u problémů, které už byly dostatečně procvičeny.
\end{solution}
