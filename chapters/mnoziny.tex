\mathquote{From the paradise that Cantor created for us no-one shall be able to expel us.}{David Hilbert}

\section{Odbočka --- Axiomatická teorie množin}
Než se pustíme do pro nás více relevatních témat, pojďme se seznámit s rigorózní teorií množin na které je postavena celá matematika. Tato teorie byla vytvořena Georgem Cantorem na konci 19. století a je základem pro všechny matematické koncepty, které se týkají množin. Tato teorie je ale velice složitá a zmiňuji ji zde jen proto, abychom měli představu o tom, jak by mohla matematika vypadat, kdybychom se snažili být formální.

Hlavní roli v této teorii hrají axiomy, což jsou základní pravidla, která musíme přijmout jako pravdivá, abychom mohli dále pracovat. Než se pustíme na samotné axiomy, vysvětleme si rychle následující znaky, které nám usnadní zápis. Množiny značíme velkými písmeny, například $X$, $Y$, $Z$, atd. Prvky množin značíme malými písmeny, například $x$, $y$, $z$, atd. Pokud prvek $x$ patří do množiny $X$, značíme to $x\in X$ a pokud nepatří, tak $x\notin X$. Systém množin je vlastně také množina, která obsahuje množiny a značíme ji $\mathcal{X}$, $\mathcal{Y}$, $\mathcal{Z}$, atd. Projděme si nyní axiomy množinové teorie a pak si ukážeme základní příklady a vysvetlení těchto axiomů. 

\begin{definitionbox}
  \textbf{Axiomy množinové teorie}
  \begin{enumerate}
    \item \textbf{Axiom of extension}: Dvě množiny jsou rovny právě tehdy, když obsahují stejné prvky.
    \item \textbf{Axiom of the empty set}: Existuje množina, která neobsahuje žádné prvky. Tato množina se nazývá prázdná množina a značí se $\emptyset$.
    \item \textbf{Axiom schema of separation}: Pro každou množinu $x$ a každou vlastnost $P$ existuje množina, která obsahuje všechny prvky množiny $x$ splňující vlastnost $P$.
    \item \textbf{Axiom of pairing}: Pro každé dvě množiny $X$ a $Y$ existuje množina, která obsahuje všechny prvky množin $X$ a $Y$. 
    \item \textbf{Axiom of union}: Pro každý systém množin $\mathcal{S}$ existuje množina, která obsahuje všechny prvky všech množin v systému $\mathcal{S}$. Tato množina se nazývá sjednocení množin v systému $\mathcal{S}$ a značí se $\bigcup \mathcal{S}$. Pokud bychom chtěli sjednotit pouze množiny $X$ a $Y$, značili bychom to $X\cup Y$.
    \item \textbf{Axiom of power set}: Pro každou množinu $X$ existuje systém všech podmnožin množiny $X$. Tento systém se nazývá potenční množina množiny $X$ a značí se $\mathcal{P}(X)$. 
    \item \textbf{Axiom of infinity}: Existuje systém množin $\mathcal{S}$ takový, že $\emptyset\in\mathcal{S}$ a pro každou množinu $x\in\mathcal{S}$ platí $x\cup\{x\}\in\mathcal{S}$. 
    \item \textbf{Axiom of choice}: Pro každý systém neprázdných množin $\mathcal{S}$ existuje funkce $f$, která každé množině $X\in\mathcal{S}$ přiřazuje prvek $f(X)$, který je prvkem množiny $X$.
    \item \textbf{Axiom schema of replacement}: Pro každou množinu $X$ a každou funkci $f$ existuje množina, která obsahuje všechny prvky $f(x)$ pro každý prvek $x\in X$.
  \end{enumerate}
  V poslední řadě chci dodat, že na přesných axiomech množinové teorie se zdroje moc neshodují a můžete najít různé formulace. Nám výše uvedené axiomy postačí pro naše účely. Pojďme si nyní podrobněji vysvětlit jednotlivé axiomy a ukázat si nějaké příklady.
\end{definitionbox}

\subsection{Axiom of extension}
K tomuto axiomu moc slov nepotřebujeme. Vezměme například tři množiny definované následovně\footnote{Všimněte si, že množiny zde zapisujeme pomocí závorek a prvky množiny oddělujeme čárkami.}:
\begin{align*}
  X &= \{1, 2, 3\},\\
  Y &= \{1, 2, 3\},\\
  Z &= \{1, 2, 3, 4\}.
\end{align*}
Množiny $X$ a $Y$ jsou rovny, protože obsahují stejné prvky, zatímco množina $Z$ je odlišná, protože obsahuje navíc prvek $4$.
\subsection{Axiom of the empty set}
I tento axiom je poměrně jednoduchý. Prázdná množina je množina, která neobsahuje žádné prvky. Můžeme ji zapsat následovně: $\emptyset = \{\}$.
\subsection{Axiom schema of separation}
Tento axiom nám říká, že pro každou množinu $x$ a každou vlastnost $P$ existuje množina, která obsahuje všechny prvky množiny $x$ splňující vlastnost $P$. Například, pokud máme množinu $X = \{1, 2, 3, 4, 5\}$ a vlastnost $P$, že prvek $x$ je sudý, můžeme vytvořit množinu $Y = \{2, 4\}$.
\subsection{Axiom of pairing}
Tento axiom nám říká, že pro každé dvě množiny $X$ a $Y$ existuje množina, která obsahuje všechny prvky množin $X$ a $Y$. Například, pokud máme množiny $X = \{1, 2\}$ a $Y = \{3, 4\}$, můžeme vytvořit množinu $Z = \{\{1, 2\}, \{3, 4\}\}$. Tento axiom využijeme při důležité definici kartézského součinu množin.
\subsection{Axiom of union}
Tento axiom nám říká, že pro každý systém množin $\mathcal{S}$ existuje množina, která obsahuje všechny prvky všech množin v systému $\mathcal{S}$. Například, pokud máme množiny $X = \{1, 2\}$ a $Y = \{2, 3\}$, můžeme vytvořit množinu $Z = X\cup Y \{1, 2, 3\}$.
\subsection{Axiom of power set}
Tento axiom nám říká, že pro každou množinu $X$ existuje systém všech podmnožin množiny $X$. Například, pokud máme množinu $X = \{1, 2\}$, můžeme vytvořit množinu $\mathcal{P}(X) = \{\emptyset, \{1\}, \{2\}, \{1, 2\}\}$. Této množině se říká potenční množina množiny $X$.
\subsection{Axiom of infinity} \label{axiom_of_infinity}
Tento axiom nám říká, že existuje systém množin $\mathcal{S}$ takový, že $\emptyset\in\mathcal{S}$ a pro každou množinu $x\in\mathcal{S}$ platí $x\cup\{x\}\in\mathcal{S}$. Tento axiom nám zaručuje existenci nekonečných množin. Pomocí tohoto axiomu můžeme definovat například množinu přirozených čísel $\mathbb{N}$ a to následovně:
\begin{align*}
  0 &= \emptyset,\\
  1 &= \{0\} = \{\emptyset\},\\
  2 &= \{0, 1\} = \{\emptyset, \{\emptyset\}\},\\
  3 &= \{0, 1, 2\} = \{\emptyset, \{\emptyset\}, \{\emptyset, \{\emptyset\}\}\},\\
  &\vdots
\end{align*}
\subsection{Axiom of choice}
Tento axiom je nejkontroverznější z celé množinové teorie, protože pokud ho pokud ho nepřijmeme, spousta zajímavých matematických konceptů se nám zhroutí. Například bez tohoto axiomu nefungují limity, protože potřebujeme vybrat prvek z nekonečné množiny posloupností a tento axiom nám zaručuje, že tam nějaký takový prvek je. K tomuto axiomu se ještě vrátíme, až se budeme zabývat konkrétními matematickými koncepty. Znamým paradoxem spojeným s tímto axiomem je Banach-Tarski paradox, který říká, že můžeme rozložit kouli na konečně mnoho koulí, které mají stejný objem jako původní koule. 
\subsection{Axiom schema of replacement}
Tento axiom nám říká, že pro každou množinu $X$ a každou funkci $f$ existuje množina, která obsahuje všechny prvky $f(x)$ pro každý prvek $x\in X$. Například, pokud máme množinu $X = \{1, 2, 3\}$ a funkci $f(x) = x^2$\footnote{Zde předpokládáme, že víte, co dělá základní operace na druhou.}, můžeme vytvořit množinu $Y = \{1, 4, 9\}$.

\section{Množiny}
Nebudeme se zde zabývat, co přímo množina je, protože bychom se pohybovali na hranici mezi matematikou a filozofií. Množinou tedy rozumíme určitou kolekci prvků, které mohou být cokoliv. Pokud jste nečetli předchozí odbočku, zopakujme si nyní základní definice a pojmy spojené s množinami.
\begin{definitionbox}
  \textbf{Značení}: Množiny značíme velkými písmeny, například $X$, $Y$, $Z$. Prvky množin značíme malými písmeny, například $x$, $y$, $z$.\\
  \textbf{Prvek množiny}: Prvek $x$ je prvkem množiny $X$, pokud $x\in X$. Opak platí, pokud $x\notin X$.\\
  \textbf{Prázdná množina}: Množina, která neobsahuje žádné prvky. Značíme ji $\emptyset$. \\
  \textbf{Podmnožina}: Množina $Y$ je podmnožinou množiny $X$, pokud každý prvek množiny $Y$ je také prvkem množiny $X$. Značíme to $Y\subseteq X$. \\
  \textbf{Rovnost množin}: Množiny $X$ a $Y$ jsou rovny, pokud obsahují stejné prvky. Značíme to $X = Y$. \\
  \textbf{Sjednocení množin}: Sjednocení množin $X$ a $Y$ je množina, která obsahuje všechny prvky množin $X$ a $Y$. Značíme to $X\cup Y$. \\
  \textbf{Průnik množin}: Průnik množin $X$ a $Y$ je množina, která obsahuje pouze ty prvky, které jsou zároveň v množinách $X$ a $Y$. Značíme to $X\cap Y$. \\
  \textbf{Rozdíl množin}: Rozdíl množin $X$ a $Y$ je množina, která obsahuje pouze ty prvky, které jsou v množině $X$ a nejsou v množině $Y$. Značíme to $X\setminus Y$. \\
  \textbf{Potenční množina}: Potenční množina množiny $X$ je množina všech podmnožin množiny $X$. Značíme to $\mathcal{P}(X)$. \\
\end{definitionbox}

% TODO obrazky?!

\begin{example}
  Mějme množiny $X = \{1, 2, 3\}$ a $Y = \{2, 3, 4\}$. Potom platí:
  \begin{align*}
    X\cup Y &= \{1, 2, 3, 4\},\\
    X\cap Y &= \{2, 3\},\\
    X\setminus Y &= \{1\},\\
    Y\setminus X &= \{4\}. \\
    \mathcal{P}(X) &= \{\emptyset, \{1\}, \{2\}, \{3\}, \{1, 2\}, \{1, 3\}, \{2, 3\}, \{1, 2, 3\}\}.
  \end{align*}
\end{example}

\begin{problem}
  Mějme množiny $X = \{1, 2, 3, 4\}$ a $Y = \{3, 4, 5\}$. Zjistěte $X\cup Y$, $X\cap Y$, $X\setminus Y$, $Y\setminus X$ a $\mathcal{P}(Y)$.
\end{problem}

\begin{solution}
  Máme:
  \begin{align*}
    X\cup Y &= \{1, 2, 3, 4, 5\},\\
    X\cap Y &= \{3, 4\},\\
    X\setminus Y &= \{1, 2\},\\
    Y\setminus X &= \{5\},\\
    \mathcal{P}(Y) &= \{\emptyset, \{3\}, \{4\}, \{5\}, \{3, 4\}, \{3, 5\}, \{4, 5\}, \{3, 4, 5\}\}.
  \end{align*}
\end{solution}

% TODO jeste sem pls zajeb nejake priklady setu

\section{Relace}
Když už nyní víme, co jsou množiny, je na čase se zabývat vztahy mezi nimi. Takovým vztahům se říká relace a jsou základem pro spoustu dalších matematických koncept. Relace můžeme chápat jako způsob, jakým jsou prvky množin propojeny. Nejprve si zopakujme základní definice a pojmy spojené s relacemi. Než si řekneme, co je to relace, musíme nejdříve definovat kartézský součin množin.

\begin{definitionbox}
  \textbf{Kartézský součin množin}: Kartézský součin množin $X$ a $Y$ je množina všech uspořádaných dvojic $(x, y)$, kde $x\in X$ a $y\in Y$. Značíme to $X\times Y$. 
\end{definitionbox}

\begin{example}
  Mějme množiny $X = \{1, 2\}$ a $Y = \{3, 4\}$. Potom platí:
  \begin{align*}
    X\times Y &= \{(1, 3), (1, 4), (2, 3), (2, 4)\}.
  \end{align*}
\end{example}

\begin{definitionbox}
  \textbf{Relace}: Relace $\mathcal R$ mezi množinami $X$ a $Y$ je podmnožina kartézského součinu $X\times Y$. Pokud máme relaci $\mathcal R$ mezi množinami $X$ a $Y$, značíme to $\mathcal R\subseteq X\times Y$. Prvek $(x, y)\in R$ značíme $x\mathcal Ry$ a říkáme, že prvek $x$ je v relaci $R$ s prvkem $y$. Je to ve zkratce jakási omezující vlastnost na kartézský součin množin. 
\end{definitionbox}

\begin{example}
  Mějme množiny $X = \{1, 2\}$ a $Y = \{3, 4\}$ a relaci $\mathcal R = \{(1, 3), (2, 4)\}$. Potom platí:
  \begin{align*}
    1\mathcal R3,\\
    2\mathcal R4.
  \end{align*}
V tomto případě máme relaci zadanou explicitně a slouží jen na ukázání poněkud zvláštního zápisu\footnote{Tento zápis, jak brzy uvidíte, moc používat nebudeme.}.
\end{example}

\begin{example}
  Nyní si pojďme ukázat příklad, kde je relace zadaná jako nějaká vlastnost, kterou prvky množin splňují. Mějme množiny $X = \{1, 2, 3\}$ a $Y = \{3, 4, 5\}$ a relaci $\mathcal R = \{(x, y)\in X\times Y\ |\ x < y\}$. Což v překladu znamená, že prvek $x$ je v relaci s prvkem $y$, pokud $x$ je menší než $y$. Potom platí:
  \begin{align*}
    1\mathcal R3,\\
    1\mathcal R4,\\
    1\mathcal R5,\\
    2\mathcal R3,\\
    2\mathcal R4,\\
    2\mathcal R5.
  \end{align*}

\end{example}

\begin{problem}
  Mějme množiny $X = \{1, 2, 3\}$ a $Y = \{3, 4, 5\}$ a relaci $\mathcal R = \{(x, y)\in X\times Y\ |\ x + y = 5\}$. Zjistěte, které dvojice prvků množin $X$ a $Y$ splňují vlastnost $x + y = 5$.
\end{problem}

\begin{solution}
  Máme:
  \begin{align*}
    1\mathcal R4,\\
    2\mathcal R3.
  \end{align*}
\end{solution}

\begin{note}
  Jak již bylo zmíněno v definice kartézského součinu, prvek $(x, y)$ je uspořádaná dvojice, což znamená, že pořadí prvků má význam. To znamená, že ve většině případů $(x, y)\neq(y, x)$. 
\end{note}

\begin{definitionbox}
  \textbf{Vlastnosti relací}. \\
    Nechť $\mathcal R$ je relace na množině $X$. Potom můžeme definovat následující vlastnosti relací: \vspace{5mm}

    \textbf{Reflexivita}: Relace $\mathcal R$ je reflexivní, pokud pro každý prvek $x\in X$ platí $x\mathcal Rx$.\ \vspace{1mm}

    \textbf{Symetrie}: Relace $\mathcal R$ je symetrická, pokud pro každý prvek $x, y\in X$ platí, že pokud $x\mathcal Ry$, tak $y\mathcal Rx$.\ \vspace{1mm}

    \textbf{Antisymetrie}: Relace $\mathcal R$ je antisymetrická, pokud pro každý prvek $x, y\in X$ platí, že pokud $x\mathcal Ry$ a $y\mathcal Rx$, tak $x = y$.\ \vspace{1mm}

    \textbf{Tranzitivita}: Relace $\mathcal R$ je tranzitivní, pokud pro každý prvek $x, y, z\in X$ platí, že pokud $x\mathcal Ry$ a $y\mathcal Rz$, tak $x\mathcal Rz$.\ \vspace{1mm}

  Pokud je relace reflexivní, symetrická a tranzitivní, říkáme, že je \textbf{relace ekvivalence}.
\end{definitionbox}

Vlastnosti relací se nejlépe ukazující na číselných množinách a proto si nejprve pojďme říct něco o nich.

\section{Čísla}
O číslech jste pravděpodobně již slyšeli, ale stejně je nutné si představit základní číselné množiny, se kterými budeme pracovat. 

\subsection{Přirozená čísla}
Přirozená čísla, jak jejich název napovídá, jsou čísla, která se nejčastěji vyskutují v našem životě. Je to také první množina, která se považovala za číselnou množinu po překvapivě dlouhou dobu.\footnote{Nula byla často rozporuplná a někteří matematikové ji neuznávali jako číslo a jejich argument byl, že není potřeba číslo, které značí nic.} 
\begin{definitionbox}
  \textbf{Přirozená čísla} jsou množina čísel, která začínají od jedničky a pokračují dál. Značíme je $\mathbb{N} = \{1, 2, 3, \ldots\}$. Pokud si vzpomínáte na odbočku o axiomech, tak konkrétně v~\ref{axiom_of_infinity} jsme si ukázali, jak můžeme přirozená čísla definovat pomocí množinové teorie.

  V případě, že chceme zahrnout nulu, mluvíme o \textbf{přirozených číslech včetně nuly} a značíme je $\mathbb{N}_0 = \{0, 1, 2, 3, \ldots\}$.
\end{definitionbox}

\begin{note}
  Sice jsme se o operacích na přirozených číslech ještě nezmínili, ale věřím, že většina z vás ví, co znamená sčítání, odčítání, násobení a dělení. Proto je nutné si uvědomit, že přirozená čísla jsou uzavřená na sčítání a násobení, ale ne na odčítání a dělení. To znamená, že pokud sečteme nebo vynásobíme dvě přirozená čísla, výsledek bude také přirozené číslo, ale pokud je odečteme nebo vydělíme, výsledek nemusí být přirozené číslo ($3-5 = -2 \not\in\mathbb{N}$ nebo $3/2 = 1.5 \not\in\mathbb{N}$).
\end{note}

\subsection{Celá čísla}
\begin{definitionbox}
  \textbf{Celá čísla} jsou množina čísel, která zahrnují přirozená čísla, ale také záporná čísla a nulu. Značíme je $\mathbb{Z} = \{\ldots, -3, -2, -1, 0, 1, 2, 3, \ldots\}$.
\end{definitionbox}

\begin{note}
  Celá čísla jsou stejně jako přirozená čísla uzavřená na sčítání a násobení a navíc na odčítání. Na dělení si budeme ještě chvíli muset počkat.
\end{note}

\begin{note}
  Další důležitá vlastnost spojena s uzavřeností na sčítání a odčítání je, že každé celé číslo má svůj opačný prvek. To znamená, že pro každé celé číslo $x$ existuje celé číslo $-x$ takové, že $x + (-x) = 0$. K tomuto faktu se ještě vrátíme, až si představíme další (a hezčí\footnote{Toto není matematický výraz, ale brzy uvidíme, co je tím myšleno.}) číselné množiny.
\end{note}

\subsection{Racionální čísla}
\begin{definitionbox}
  \textbf{Racionální čísla} jsou množina čísel, která zahrnují celá čísla, ale také zlomky. Značíme je $\mathbb{Q} = \left\{\frac{a}{b}\ |\ a, b\in\mathbb{Z}, b\neq 0\right\}$.\ \vspace{2mm}

  Tato definice je trochu složitější, ale znamená to, že racionální čísla jsou všechna čísla, která můžeme zapsat jako zlomek, kde čitatel i jmenovatel jsou celá čísla a jmenovatel není nula. 
\end{definitionbox}

\begin{note}
  Racionální čísla jsou uzavřená na sčítání, odčítání, násobení a dělení, pokud dělitel není nula. To znamená, že pokud sečteme, odečteme, vynásobíme nebo vydělíme dvě racionální čísla, výsledek bude také racionální číslo. Tento fakt je důležitý, protože nám říká, že racionální čísla tvoří těleso\footnote{Těleso je matematický pojem, který znamená, že množina je uzavřená na sčítání, odčítání, násobení a dělení.}.
\end{note}